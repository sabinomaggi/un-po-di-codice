\documentclass[11pt,oneside,a4paper]{report}
\usepackage{framed}

\begin{document}

\section*{Ultimi post su Melabit}

\begin{framed}
\begin{description}
    \item[Tipo:] post
    \item[Titolo:] "Il CNR è anche questo: un po' di codice"
    \item[Autore:] Sabino Maggi
    \item[Data:] 2020-12-01 06:00
    \item[Categorie:] scienza
    \item[Tag:] awk, github, latex, linux, macos, r, rstudio desktop, terminale, windows, xkcd
    \item[Commenti:] true
    \item[Pubblicato:] true
    \item[Testo iniziale del post:] 
    ![](https://imgs.xkcd.com/comics/goodcode.png "XKCD, Good code")
    - XKCD, [Good code](https://xkcd.com/844/).
    Per concludere nel miglior modo possibile questa serie di articoli (qui la
    [prima](https://melabit.wordpress.com/2020/10/25/il-cnr-e-anche-questo-concorsi-senza-
    frontiere/) e la [seconda](https://melabit.wordpress.com/2020/11/08/il-cnr-e-anche-questo-
    concorsi-in-latex/) parte), cosa ci può essere di meglio di un po' di codice?
\end{description}
\end{framed}

\begin{framed}
\begin{description}
    \item[Tipo:] post
    \item[Titolo:] "Il CNR è anche questo: concorsi in LaTeX"
    \item[Autore:] Sabino Maggi
    \item[Data:] 2020-11-08 06:00
    \item[Categorie:] scienza
    \item[Tag:] cnr, concorso, git, latex, r, ricercatori, terminale, word
    \item[Commenti:] true
\end{description}
\end{framed}

\begin{framed}
\begin{description}
    \item[Tipo:] post
    \item[Titolo:] "Il CNR è anche questo: concorsi senza frontiere"
    \item[Autore:] Sabino Maggi
    \item[Data:] 2020-10-25 06:00
    \item[Categorie:] scienza
    \item[Tag:] cnr, concorso, curriculum, fattispecie, ricercatori, word
    \item[Commenti:] true
    \item[Pubblicato:] true
    \item[Testo iniziale del post:] 
    https://www.youtube.com/watch?v=T8el-AHcNFs
    Lo so, è da un mese che non pubblico niente sul blog. Non che mi manchino gli spunti,
    tutt'altro, la lista di cose da scrivere si allunga ogni giorno di più. Ma è dall'inizio di agosto
    che mi trovo ad affrontare una scadenza importante dopo l'altra, e appena ne supero una
    eccone un'altra ancora più pressante. Dopo giornate come quelle delle ultime settimane, è
    difficile mettersi di nuovo alla scrivania la sera per aggiornare il blog.
\end{description}
\end{framed}

\begin{framed}
\begin{description}
    \item[Tipo:] post
    \item[Titolo:] "Tech porn"
    \item[Autore:] Sabino Maggi
    \item[Data:] 2020-09-16 06:00
    \item[Categorie:] scienza
    \item[Tag:] apple, laboratorio, nasa
    \item[Commenti:] true
    \item[Pubblicato:] true
    \item[Testo iniziale del post:] 
    Da ragazzo, oltre che su Playboy, sbavavo su immagini come questa, e sognavo di poter
    usare un giorno uno di questi strumenti complicatissimi.
    ![](https://melabit.files.wordpress.com/2020/09/liebergot-8210-640x426-1.jpg)
    -- Foto [ArsTechnica](https://arstechnica.com/science/2012/10/going-boldly-what-it-was-
    like-to-be-an-apollo-flight-controller/).
\end{description}
\end{framed}

\begin{framed}
\begin{description}
    \item[Tipo:] post
    \item[Titolo:] "Il (mio) gioco dell'estate"
    \item[Autore:] Sabino Maggi
    \item[Data:] 2020-08-21 06:00
    \item[Categorie:] software
    \item[Tag:] colorzzle, ios, ipad, monument valley
    \item[Commenti:] true
    \item[Pubblicato:] true
    \item[Testo iniziale del post:] 
    Non sono un gran giocatore, però mi piace giocare con l'iPad la sera prima di dormire
    oppure in vacanza in estate. Niente di complicato, i giochi con troppe regole mi annoiano,
    ancora di più lo fanno i platform, dove devi ripetere sempre gli stessi movimenti nella
    sequenza corretta.
\end{description}
\end{framed}

\begin{framed}
\begin{description}
    \item[Tipo:] post
    \item[Titolo:] "SSD o dischi meccanici? Perché non tutti e due? (seconda parte)"
    \item[Autore:] Sabino Maggi
    \item[Data:] 2020-08-14 06:00
    \item[Categorie:] hardware
    \item[Tag:] apple, hard disk, imac, imac pro, mac mini, mac pro, macbook, macos, pc, ssd, windows 10
    \item[Commenti:] true
    \item[Pubblicato:] true
    \item[Testo iniziale del post:] 
    ![](https://melabit.files.wordpress.com/2020/08/edgar-chaparro-d9uqsghl2ug-unsplash.jpg)
    -- Foto di [Edgar Chaparro](https://unsplash.com/@echaparro) su
    [Unsplash](https://unsplash.com).
    Dopo aver mostrato nella [prima parte
    dell'articolo](https://melabit.wordpress.com/2020/08/05/ssd-o-dischi-meccanici-perche-
    non-tutti-e-due-prima-parte/) perché è conveniente utilizzare un disco SSD relativamente
    piccolo dove installare il sistema operativo e le applicazioni, accoppiato ad un hard-disk
    meccanico (HDD) molto più capiente riservato ai documenti,[1] in questa seconda parte
    presenterò alcuni scenari pratici di applicazione di questa tecnica.
\end{description}
\end{framed}

\begin{framed}
\begin{description}
    \item[Tipo:] post
    \item[Titolo:] "SSD o dischi meccanici? Perché non tutti e due? (prima parte)"
    \item[Autore:] Sabino Maggi
    \item[Data:] 2020-08-05 06:00
    \item[Categorie:] hardware
    \item[Tag:] apple, hard disk, imac, linux, macos, mac pro, pc, ssd, windows
    \item[Commenti:] true
    \item[Pubblicato:] true
    \item[Testo iniziale del post:] 
    ![](https://melabit.files.wordpress.com/2020/08/laura-ockel-qox9ksvpqcm-unsplash.jpg)
\end{description}
\end{framed}

\end{document}
